\categorytitle{Combinatorial}
\categorycontents{}

\problemtitle{Sorting}
\begin{algorithm}{sort}
  \usage{\sourceline{ sort( v.begin(), v.end(), less<int>() ) }}
\end{algorithm}

\begin{algorithm}{isort}
  \usage{\sourceline{ isort( array, n, idxarray, comp) }}
  \reflisting{isort}
\end{algorithm}

\begin{algorithm}{indexed comparator}
  \reflisting{indexed comparator}
\end{algorithm}

\begin{algorithm}{indexed less}
  \reflisting{indexed less}
\end{algorithm}

\problemtitle{Searching}
\begin{algorithm}{median -- nth element}
  \usage{\sourceline{ nth\_element( v.begin(), v.end(), 4 ) }}
  \begin{example}
	nth_element( v.begin(), v.end(), v.size() / 2 ) // mid value
  \end{example}
\end{algorithm}
\begin{algorithm}{binary search}
  \usage{\sourceline{ bool binary\_search( v.begin(), v.end(), 4 ) }}
  \usage{\sourceline{ iterator lower\_bound( v.begin(), v.end(), 4 ) }}
  \usage{\sourceline{ iterator upper\_bound( v.begin(), v.end(), 4 ) }}
\end{algorithm}
\begin{algorithm}{binary search (numerical)}
  \reflisting{binary search num}
  \usage{\sourceline{double binary\_search\_num( 0.0, 1.0, pred ); }}
\end{algorithm}
\begin{algorithm}{golden search}
  \reflisting{golden search}
  \usage{\sourceline{
 int golden\_search( f, l, r, \&min\_value ) == min\_index }}
\end{algorithm}

\problemtitle{Permutations}
\begin{algorithm}{next}
\usage{{\tt next\_permutation(begin, end [, comparator])}}
\end{algorithm}
\begin{algorithm}{previous}
\usage{{\tt prev\_permutation(begin, end [, comparator])}}
\end{algorithm}
\begin{algorithm}{random}
\usage{{\tt random\_shuffle(begin, end [, random\_number\_generator])}}
\end{algorithm}
\begin{algorithm}{permute}
\reflisting{permute}
\desc
{\tt permute} recursively generates all permutations
without comparisons. The permutations are given in lexicographic order
using the original order (i.e. permute on ``baa'' gives baa, aba, aab).
It can also be useful when the processing is done for each individual
position (unnecessary duplicate work is avoided). Repetitions of non-adjacent
elements will give strange results since there is no possible ordering
(i.e ``aba'').
\usage{\sourceline{char a[n] = "caab"; permute(a+0,a+0,a+n);}}
\end{algorithm}
\begin{algorithm}{Permutations to/from integers}
\reflisting{intperm}
\usage{{\tt int perm[n], x;\\perm\_to\_int(x, perm, perm + n);\\int\_to\_perm(x, perm, perm + n);}}
\complexity{ n^2 }
\desc
{\tt int\_to\_perm} and {\tt perm\_to\_int} defines a bijection
between permutations $\pi \in S_n$ and integers $x \in \{0, \ldots
n!-1\}$, with the additional property that it is order-preserving (so
that it could be used for generating all permutations in lexicographic
order, though {\tt next\_permutation} is much handier for this).  In
theory, this could be useful when doing some graph search on
permutations (though in practice, the number of integers for which
$n!$ but not $n^n$ memory is feasible is quite small (2-3 depending on
what you consider feasible)).
\end{algorithm}

\xinput{counting}

\begin{sourceslandscape}
\sourcesection{Sorting \& Searching}
\code{isort}{isort}
\code{indexed comparator}{indexed_comparator}
\code{indexed less}{indexed_less}
\code{binary search num}{binary_search_num}
\code{golden search}{golden_search}

\sourcesection{Permutations \& Counting}
\code{permute}{permute}
\code{intperm}{intperm}
\code{choose}{counting/choose}
\code{multinomial}{counting/multinomial}
\code{nperms}{counting/nperms}
\code{stirling1}{counting/stirling1}
\code{stirling}{counting/stirling}
\code{stirling mod 2}{counting/smod2}
\code{euler}{counting/euler}
\code{euler2}{counting/euler2}
\end{sourceslandscape}
