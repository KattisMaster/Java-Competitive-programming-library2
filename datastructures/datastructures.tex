\categorytitle{Data Structures}
\categorycontents{}
\problemtitle{pair}
\problemtitle{map}
\problemtitle{set}
\problemtitle{queue}
\problemtitle{deque}

\problemtitle{priority queue}
\begin{algorithm}{mpq}
\reflisting{mpq}

{\tt mpq} is a modifiable priority queue (implemented as a set). Its interface
is identical to that of a {\tt priority\_queue}. When an element should be
modified the {\tt update} method should be called as:
\sourceline{update(elem, oldvalue, newvalue)}
Where {\tt oldvalue} should be a \emph{reference} to the value of the
{\tt elem}.

A common use is to use indices as elements which is compared using external
containers.
\end{algorithm}

\problemtitle{stack}
\problemtitle{vector}
\problemtitle{list}
\problemtitle{string}

\problemtitle{heap}
\begin{algorithm}{update heap}
\reflisting{update heap}

An updatable heap has an interface identical to that of a
{\tt priority\_queue}.
The elements need to have a method {\tt set\_position} though. When an element
is changed, the {\tt key\_increased} or {\tt key\_decreased} method should
be called with its position as argument.
\end{algorithm}

\problemtitle{valarray}

\begin{sourceslandscape}
\code{update heap}{update_heap}
\code{mpq}{mpq}
\end{sourceslandscape}
