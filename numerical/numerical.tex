\clearpage
\categorytitle{Numerical}
\categorycontents{}

\problemtitle{Number theory}

\code{euclid}{euclid}

\code{chinese}{chinese}

\begin{algorithm}{Primes}
\desc
The 1000th prime is 7919. The first every 10000th primes are:
{\small
\begin{verbatim}
  104729  224737  350377  479909  611953
  746773  882377 1020379 1159523
\end{verbatim}
}\end{algorithm}

\code{prime sieve}{prime_sieve}

\code{miller-rabin}{miller-rabin}

\code{pollard-rho}{pollard-rho}

\begin{algorithm}{Perfect numbers}
\desc
$n$ is perfect iff $n = \frac{p(p+1)}{2}$, where $p = 2^k-1$ is prime.
First Mersenne primes are obtained for $k =$ 2, 3, 5, 7, 13, 17, 19,
31, 61, 89, 107, 127, 521, 607, 1279, 2203, 2281, 3217, 4253, 4423,
9689, 9941, 11213, 19937, 21701, 23209, 44497.
\end{algorithm}

\begin{algorithm}{Josephus}
\desc
Which person remains when repeatedly removing the $k$:th person from a
total of $n$ persons (cyclic)?
\complexity{\log_{\frac{k}{k-1}}(n)}
\end{algorithm}

\code{josephus}{josephus}

\problemtitle{Linear Equations}
\code{solve linear}{solve_linear}
\code{matrix inverse}{matrix_inverse}

\begin{algorithm}{Calculating determinant}
\desc {\tt determinant} and {\tt int\_determinant} both reduces the matrix
to an upper diagonal form using elementary row operations. There could be an
overflow in the integral variant and in that case the double variant
can be used instead, rounding the answer at the end. The strength of
{\tt int\_determinant} is that it can be used for {\tt long long} or
{\tt BigInt}. Note that it uses {\tt euclid} which could be rather
slow in the {\tt BigInt} case.
\end{algorithm}
\code{determinant}{determinant}
\code{int determinant}{int_determinant}

\problemtitle{Optimization}
\begin{algorithm}{Simplex method}
\desc
Solves a linear minimization problem. The first row of the
input matrix is the objective function to be minimized. The
first column is the maximum allowed value for each linear row.
\end{algorithm}
\code{simplex}{simplex}

\problemtitle{Polynomials}
\code{polynomial}{polynomial}
\code{poly roots}{poly_roots}

\problemtitle{Bit manipulation hacks}
\code{bitmanip}{bitmanip}


