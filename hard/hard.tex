\categorytitle{Hard problems}
\categorycontents{}

\problemtitle{Knapsack}


\begin{algorithm}{Knapsack}
\reflisting{knapsack}
\usage{{\tt R res = knapsack<R>(n, C, costs, values [, bound = 500000]);}}
\complexity{\min(bound, nC)}

\desc
Knapsack heuristic. Returns the maximum value achievable. {\tt n} is
the number of objects, {\tt C} is the capacity of the knapsack, {\tt
costs} is a random access container with the {\tt n} costs, and {\tt
values} is a random access container with the {\tt n} values. {\tt
bound} is an approximation factor; lower values of {\tt bound} means
shorter running time, but also a greater risk that the algorithm will
produce an incorrect answer (if $nC \le $ {\tt bound} and the costs
are integers, the answer is ``guaranteed'' to be correct). Note that
actually, all scaled values (i.e. multiplications with {\tt scale})
should be rounded upwards. However, empirical tests (using the test
data from NADA Open 2002) has shown this approach to be less accurate
(i.e. requiring a higher bound to produce a correct answer) in
practice. For the NADA Open 2002 test cases, a bound of $\approx
90000$ was sufficient.

\end{algorithm}


\begin{sourceslandscape}
\code{knapsack}{knapsack}
\end{sourceslandscape}
