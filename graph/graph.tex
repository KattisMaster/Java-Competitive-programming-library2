\categorytitle{Graph}
\categorycontents{}

\problemtitle{Connected components}
\problemtitle{Flood fill}
\xinput{toposort}
\xinput{shortest_path}
\xinput{mst}
\problemtitle{Transitive Closure}
The transitive closure, i.e. which nodes are connected, is found using
a sligthly modified Floyd-Warshall. The algorithm also gives a path between
two nodes, if they are connected.

\begin{algorithm}{Transitive Closure}
\reflisting{transitive closure}
\complexity{V^3}
\usage{ \sourceline{ floyd\_warshall(adj, path,n); } }

The algorithm uses a bool adjacency matrix. The adjacency matrix should contain
wheter there exists a path between {\tt a} and {\tt b} in {\tt adj[a][b]}.
The adjacency matrix is updated to contain whether {\tt a} and
{\tt b} are connected and {\tt path[a][b]} contains the from-node for a
path between {\tt a} and {\tt b}.
\end{algorithm}

\problemtitle{Euler Cycle and Chinese Postman}
\problemtitle{Edge and Vertex Connectivity}
\xinput{flow}
\problemtitle{Matching}
\problemtitle{Planarity detection}

\begin{sourceslandscape}
\code{topo sort}{topo_sort}
\code{floyd warshall}{floyd_warshall}
\code{dijkstra}{dijkstra}
\code{dijkstra 1}{dijkstra_1}
\code{dijkstra heap}{4_dijkstra_heap}
\code{dijkstra prim}{dijkstra_prim}
\code{get shortest path}{5_get_shortest_path}
\code{transitive closure}{transitive_closure}
\code{sets}{mst/sets}
\code{kruskal}{mst/kruskal}
\code{euler walk}{6_euler_walk}
\code{flow graph}{maxflow/flow_graph}
\code{ford fulkerson}{maxflow/ford_fulkerson}
\code{ford fulkerson 1}{maxflow/ford_fulkerson_1}
\code{lift to front}{maxflow/lift_to_front}
\end{sourceslandscape}
