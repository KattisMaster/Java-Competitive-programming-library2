\problemtitle{Network Flow}

Flow graphs are directed graphs with flow capacities on their edges.

Many problems can be solved by finding the maximum flow (or, equivalently,
the minimal cut) of a flow graph.

The flow algorithms need quick access to the ``back edge'' of all egdes.
To accomodate for this, a special flow edge struct is used.
Flow graphs are constructed and updated by a couple of utility functions.

A flow graph is given as a STL-container of {\tt vector}s ({\tt map}s are
not allowed) with edges. The edges are {\tt flow\_edge}'s and must be added
using {\tt flow\_add\_edge}. Note that a edge {\emph must be} added only once
for each pair, simultaneously giving both forward and back capacity.


\begin{algorithm}{flow graph}
\reflisting{flow graph}
\usage{ \sourceline{flow\_add\_edge( edges, source, dest, cap [, back\_cap] );}
}
\end{algorithm}

\begin{algorithm}{ford fulkerson}
\reflisting{ford fulkerson}
\characteristics{This is a DFS or BFS Ford Fulkerson which maximize the
flow in the augmenting paths. The BFS is more robust but may be slower.}
\usage{ The maximum flow is calculated by repetitive calls
to {\tt flow\_increase1}:}

\sourceline{ while( ap = flow\_increase1(edges, source, sink) ) flow+=ap; }
\complexity{E\cdot n_{aug~paths}}
\end{algorithm}

\begin{algorithm}{ford fulkerson 1}
\reflisting{ford fulkerson 1}
\characteristics{This is a DFS Ford Fulkerson where all augmenting paths
are 1 and thus specially suited for bipartite graphs.}
\usage{ The maximum flow is calculated by repetitive calls
to {\tt flow\_increase1}: }

\sourceline{ while( flow\_increase1(edges, source, sink) ) flow++; }
\complexity{E\cdot n_{aug~paths}}
\end{algorithm}

\begin{algorithm}{lift to front}
\reflisting{lift to front}
\characteristics{This is a much more effective algorithm than Ford Fulkerson,
even on bi-partite graphs, and suitable for any flow graph.}
\usage{ \sourceline{flow = lift\_to\_front(edges, source, sink);} }
\end{algorithm}
