\clearpage
\categorytitle{Combinatorial}
\categorycontents{}


\problemtitle{Misc}

\begin{algorithm}{Impartial take-and-break games (NIM-like games)}
\desc
A game where two players take turns removing (indistinguisable) tokens
from some heaps of tokens.  The player removing the last token (thus
causing the next player to be unable to move) is the winner.  The
moves available are: removing $x$ tokens from a heap (for some set of
allowed $x$), and splitting a heap of $n$ tokens into two heaps of
$n_1$ and $n_2$ tokens where $n_1, n_2 < n$.  Because every move
reduces a heap size by at least 1, such games can never end in draw.
To find optimal strategies, Grundy numbers (or nimbers) can be used.
The Grundy value of a state $S = \{n\}$ is defined as $G(S) =
\mathrm{mex}\;S'$ where $S'$ runs over all successor states to $S$ and
$\mathrm{mex}$ is the minimal excluded (nonnegative) value.  The
Grundy value of $S =
\{n_1, n_2, \ldots n_k\}$ is defined as $\bigoplus_{i=1}^{k}
G(\{n_i\})$.  A state $S$ is winning iff $G(S) \ne 0$.

\end{algorithm}


\begin{algorithm}{Knapsack}
\complexity{\min(bound, nC)}
\end{algorithm}
\codenc{knapsack}{knapsack}

\problemtitle{Puzzles}

\code{nqueens}{nqueens}
\code{magicsquare}{magicsquare}

\begin{algorithm}{Tournament scheduling}
\desc
Compute a minimal day schedule where all teams meet all other
teams. {\tt sched[i][j]} will contain the opponent of team {\tt j} on
day {\tt i}, or 0 if no game that day.  Returns number of days.
\end{algorithm}
\codenc{tournaments}{tournaments}


\problemtitle{Permutations}
\begin{algorithm}{Permutations to/from integers}
\complexity{ n^2 }
\note{The bijection is order-preserving}
\end{algorithm}
\code{intperm}{intperm}


\xinput{counting}

