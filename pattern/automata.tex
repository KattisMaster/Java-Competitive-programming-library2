\problemtitle{Automata}

\begin{algorithm}{NFA}
\reflisting{nfa}
\note{The primary use of the NFA in its current state is for RegExp matching.}
\end{algorithm}

\begin{algorithm}{DFA}
\reflisting{dfa}
\desc
A DFA class implementation. Biggest feature is the possibility to create a
DFA from an NFA, which provides for faster matching for automata that will
run several times.

No DFA compression is made. This means, for example, that the regexp
{\tt (a|b|c|d|e|f)*} will result in a DFA with seven states, occupying
roughly 7 $\cdot$ 256 bytes of memory, (whereas the optimal number of states
is 1, occupying roughly 256 bytes of memory). The equivalent regexp
{\tt [abcdef]*} will result in a DFA of 2 states.
\end{algorithm}
