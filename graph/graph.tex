\clearpage
\categorytitle{Graph}
\categorycontents{}

\problemtitle{Fundamentals}

\begin{algorithm}{Bellman-Ford}
\complexity{VE}
\end{algorithm}
\code{bellman ford}{bellman_ford}


\begin{algorithm}{Shortest Tour}
\desc
Shortest tour from A to B to A again not using any edge twice, in an
undirected graph: Convert the graph to a directed graph.  Take the
shortest path from A to B.  Remove the paths used from A to B, but
also negate the lengths of the reverse edges.  Take the shortest path
again from A to B, using an algorithm which can handle negative-weight
edges, such as Bellman-Ford. Note that there is no negative-weight
\emph{cycles}.  The shortest tour has the length of the two shortest
paths combined. 
\end{algorithm}


\begin{algorithm}{Kruskal}
\usage{ \sourceline{kruskal( graph, tree, n );} }
\complexity{ E\log E }
\desc
{\bf NB!} Requires {\texttt sets.cc}!
The resulting tree which is returned in {\tt tree} may
be the same variable as the graph.
\reflisting{sets}
\end{algorithm}
\codenc{kruskal}{mst/kruskal}

\begin{algorithm}{Directed Minimum Spanning Tree}
\desc
Chu-Liu/Edmonds Algorithm:
\begin{enumerate}

\item Discard the arcs entering the root if any; For each node other than the root, select the entering arc with the smallest cost; Let the selected $n-1$ arcs be the set $S$.

\item  If no cycle formed, $G(N,S)$ is a MST. Otherwise, continue.  


\item For each cycle formed, contract the nodes in the cycle into a pseudo-node
 $(k)$, and modify the cost of each arc which enters a node $(j)$ in the
  cycle from some node $(i)$ outside the cycle according to the
  following equation:  $$c(i,k)=c(i,j)-(c(x(j),j)-min_{j}(c(x(j),j))$$
  where $c(x(j),j)$ is the cost of the arc in the cycle which enters $j$.
  
\item For each pseudo-node, select the entering arc which has the smallest modified cost; Replace the arc which enters the same real node in $S$ by the new selected arc.

\item Go to step 2 with the contracted graph.
\end{enumerate}
\end{algorithm}

\code{topo sort}{topo_sort}



\xinput{euler_cycle}
\xinput{network_flow}
\xinput{matching}
%\problemtitle{Planarity detection}
