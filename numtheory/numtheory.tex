\clearpage
\categorytitle{Number theory}
\categorycontents{}

\problemtitle{Primality}

\begin{algorithm}{Primes}
\desc
1000th prime is 7919. First every 10000th primes are: {\small
\begin{verbatim}
  104729  224737  350377  479909  611953
  746773  882377 1020379 1159523
\end{verbatim}
}\end{algorithm}

\code{prime sieve}{prime_sieve}

\code{miller-rabin}{miller-rabin}

\problemtitle{Divisibility}

\code{euclid}{euclid}

\begin{algorithm}{Chinese remainder theorem}
\desc
Solves the system $x = a \pmod m$, $x = b \pmod n$.  {\tt chinese}
returns the unique solution with $0 \le x < \mathrm{lcm}(m, n)$.  If
$\gcd(m,n) = 1$, {\tt chinese} may be used, otherwise, {\tt
chinese\_common} must be used, which returns $-1$ if there is no
solution.
\end{algorithm}
\codenc{chinese}{chinese}

\code{pollard-rho}{pollard-rho}

\begin{algorithm}{Perfect numbers}
\label{perfnum}
\desc
$n$ is perfect iff $n = \frac{p(p+1)}{2}$, where $p = 2^k-1$ is prime.
First Mersenne primes are obtained for $k =$ 2, 3, 5, 7, 13, 17, 19,
31, 61, 89, 107, 127, 521, 607, 1279, 2203, 2281, 3217, 4253, 4423,
9689, 9941, 11213, 19937, 21701.\\
\end{algorithm}

