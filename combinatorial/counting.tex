%%%
%%% Combinatorics
%%%

\problemtitle{Counting}

\begin{algorithm}{Binomial $\binom{n}{k}$}
\complexity{\min\{k, n-k\}}
\end{algorithm}
\code{choose}{counting/choose}

\begin{algorithm}{Multinomial $\binom{\Sigma k_i}{k_1\;k_2\;\ldots\;k_n}$}
\complexity{(\Sigma k_i)-k_1}
\end{algorithm}
\code{multinomial}{counting/multinomial}

\begin{algorithm}{Partitions}
\desc
The number of partitions of $n$ non-distinct elements is
$$p(n) = \sum_{k \ge 1} (-1)^{k+1} \left[p(n- k(3k-1)/2) + p(n-k(3k+1)/2)\right]$$
for $n \ge 1$, and $p(n) = [n == 0]$ for $n < 1$.

The number of partitions into distinct, odd parts is
$$q(n) = \frac{1}{n} \sum_{k = 1}^{n} (-1)^{k+1} \sigma(k) q(n-k)$$
where $\sigma(k)$ is the sum of the {\em odd} divisors of $k$
(e.g. $\sigma(9) = 13$).  If $(-1)^{k+1}$ is removed, we get the
number of partitions into distinct parts.

\end{algorithm}

\begin{algorithm}{Stirling numbers of the first kind}
\desc
$s(n,k)$ is defined as
$(-1)^{n-k}c(n,k)$, where $c(n,k)$ is the number of permutations on
$n$ items with $k$ cycles.
$$s_{n,k} = s_{n-1,k-1} - (n-1)s_{n-1,k}$$
$$s_{n,k} = 1, n = k \qquad s_{n,k} = 0, n < 1$$
\end{algorithm}

\begin{algorithm}{Stirling numbers of the second kind}
\desc
Partitions of $n$ elements in exactly $k$ boxes.
$$s_{n,k} = s_{n-1,k-1} + ks_{n-1,k}$$
$$s_{n,k} = 1, n = k \qquad s_{n,k} = 0, n < 1$$
\end{algorithm}

\begin{algorithm}{Bell numbers}
\desc
$B(n) = \sum_{k=1}^n \binom{n-1}{k-1} B(n-k) = \sum_{k=1}^n S(n,k)$,
where $S(n, k)$ are the Stirling numbers of the second kind.

The Bell numbers count the ways $n$ distinct elements can be partitioned.
\end{algorithm}

\begin{algorithm}{Eulerian numbers}
\desc
The Eulerian number $e_{n,k}$ is the number of $\pi \in S_n$ with
\begin{itemize}
\item $k$ $j$:s s.t. $\pi(j) > \pi(j+1)$
\item $k+1$ $j$:s s.t. $\pi(j) \ge j$
\item $k$ $j$:s s.t. $\pi(j) > j$
\end{itemize}
\begin{eqnarray*}
e_{n,k} & = & (n-k)e_{n-1,k-1} + (k+1) e_{n-1, k}\\
        & = & \sum_{j=0}^{k+1}(-1)^j \binom{n+1}{j} (k - j + 1)^n
\end{eqnarray*}
$$e_{n,k} = 1, n = k = 0 \qquad e_{n,k} = 0, n < 1 \vee n = k \ne 0$$
\end{algorithm}

\begin{algorithm}{Second-order Eulerian numbers}
\desc
The second-order Eulerian number $e_{nk}$ is the number of
permutations $\pi_1 \pi_2 \cdots \pi_{2n}$ of the multiset
$\{1,1,2,2,\cdots,n,n\}$ with the property that all numbers between
the two occurences of $m$ are greater than $m$ that have $k$ places
where $\pi_j < \pi_{j+1}$. It is given by
$$e_{n,k} = (2n-1-k)e_{n-1,k-1} + (k+1) e_{n-1, k}$$
$$e_{n,k} = 1, n = k = 0 \qquad e_{n,k} = 0, n < 1 \vee n = k \ne 0$$
\end{algorithm}

\begin{algorithm}{Catalan numbers}
\desc
$$C_n = \frac{2(2n-1)C_{n-1}}{n+1} = \frac{\binom{2n}{n}}{\scriptstyle n+1}$$
\end{algorithm}

\begin{algorithm}{Derangements}
\desc
\begin{eqnarray*}
D_n & = & (n-1)(D_{n-1}+D_{n-2}) = nD_{n-1} + (-1)^n \\
    & = & n!\left(\frac 1{2!}-\frac 1{3!}+\ldots+(-1)^n\frac 1{n!}\right) =
\left\lfloor\frac{n!}{e}\right\rceil
\end{eqnarray*}
{\tt derangements.cc} generates the $i$:th (lexicographically)
derangement of $S_n$.
\end{algorithm}
\codenc{derangements}{derangements}

\begin{algorithm}{Involutions}
\desc
An involution is a permutation with maximum cycle length 2, or
equivalently, a permutation which is its own inverse.  The number of
involutions on $[n]$ is given by
$$s(n) = s(n-1) + (n-1)s(n-2) \qquad s(0) = s(1) = 1$$
\end{algorithm}

\begin{algorithm}{Triangles}
\desc
Given rods of length $1, \ldots, n$, 
$$
T(n) = \frac{1}{24}\left\{\begin{array}{ll}
n(n-2)(2n-5) & \textrm{$n$ even}\\
(n-1)(n-3)(2n-1) & \textrm{$n$ odd}
\end{array}\right.$$
distinct triangles (positive area) can be constructed.
\end{algorithm}

\begin{algorithm}{Motzkin Numbers}
\desc
Counts for instance: number of ways of drawing any number of
nonintersecting chords among $n$ points on a circle, number of lattice
paths from $(0,0)$ to $(n,0)$ never going below the $x$ axis and using
only steps NE, E and SE, number of $(3412,2413)$-, $(3412,3142)$-, and
$(3412,3412)$-avoiding involutions in $S_n$.
$$
M_n = \frac{3(n-1)M_{n-2} + (2n+1)M_{n-1}}{n+2}
$$
with $M_0 = M_1 = 1$.
\end{algorithm}

\begin{algorithm}{Married Couples Problem}
\desc
In how many ways can $n$ couples be seated around a round table such
that men and women are alternated and such that no couple sits next to
each other?
$$a_n = \frac{(n^2-2n)*a_{n-1} + n*a{n-2} - 4(-1)^n}{n-2}$$
for $n \ge 4$, with $a_0 = a_3 = 1$, $a_1 = a_2 = 0$.

\end{algorithm}