\clearpage
\categorytitle{Useful mathematical identities}
\categorycontents{}

\problemtitle{Equations}

$$
\begin{array}{c@{+}c@{=}c}
ax & by & e \\
cx & dy & f
\end{array} \Rightarrow
\begin{array}{c@{=}c}
x & \frac{ed-bf}{ad-bc} \\
y & \frac{af-ec}{ad-bc}
\end{array}
$$

\problemtitle{Trigonometry}

\begin{align*}
  \sin(v+w) &= \sin v \cos w + \cos v \sin w \\
  \cos(v+w) &= \cos v \cos w  - \sin v \sin w  \\
  \tan(v+w) &= \frac{\tan v +\tan w }{1-\tan v \tan w }\\
  \sin v +\sin w  &= 2\sin\frac{v+w}{2}\cos\frac{v-w}{2}\\
  \cos v + \cos w & = 2 \cos \frac{v+w}{2} \cos \frac{v-w}{2}\\
  (V+W)\tan (v-w)/2 &= (V-W)\tan (v+w)/2
\end{align*}
where $V$ is the length of the side opposite the angle $v$, and
similar for $W$.
$$
\left\{\begin{array}{rcl}
a\cos x + b \sin x &=& r\cos(x - \phi)\\
a\sin x + b \cos x &=& r\sin(x + \phi)
\end{array}\right.
$$
where $r = \sqrt{a^2+b^2}$, $\phi = \arctan(b,a)$.
\begin{algorithm}{Spherical trigonometry}
\desc
$a,b,c$ = sides, $\alpha, \beta, \gamma$ = angles, all six
anglemeasured and less than $\pi$.
\begin{align*}
  \cos a & = \cos b \cos c + \sin b \sin c \cos \alpha\\
  \cos \alpha & = -\cos \beta \cos \gamma + \sin \beta \sin \gamma \cos a\\
  \sin \alpha / \sin a & = \sin \beta / \sin b = \sin \gamma / \sin c\\
\end{align*}
\end{algorithm}

\problemtitle{Geometry}
\begin{algorithm}{Triangles}
\desc
Side lengths $a, b, c$.\\
Semiperimeter $p =\frac{a+b+c}{2}$.\\
Area $A = \sqrt{p(p-a)(p-b)(p-c)}$.\\
Circumradius $R = \frac{abc}{4A}$.\\
Inradius $r = \frac{A}{p}$.\\
Median (divides triangle into two equal-sized triangles) $m_a =
\frac{1}{2}\sqrt{2b^2+2c^2-a^2}$.\\
Bisector (divides angle in two) $s_a = \sqrt{bc\left[1-\left(\frac{a}{b+c}\right)^2\right]}$.
$$\frac{\sin \alpha}{a} = \frac{\sin \beta}{b} = \frac{\sin \gamma}{c} = \frac{1}{2R}$$
$$a^2 = b^2 + c^2 - 2bc\cos \alpha$$
$$\frac{a+b}{a-b}= \frac{\tan\frac{\alpha+\beta}{2}}{\tan\frac{\alpha-\beta}{2}}$$
\end{algorithm}

\begin{algorithm}{Quadrilaterals}
\desc
Side lengths $a, b, c, d$.\\
Diagonals $e (ad \leftrightarrow bc), f (ab \leftrightarrow cd)$.\\
Diagonal angle $\theta$.\\
Magic flux $F = b^2+d^2-a^2-c^2$.\\
Area $4A = 2ef\sin \theta = F\tan \theta = \sqrt{4e^2f^2 - F^2}$.\\

\end{algorithm}

\begin{algorithm}{Regular Polyhedra}
\desc
Vertices $v$.\\
Edges $e$.
Faces $f = e-v+2$.
Volume $V = k_V a^3$.\\
Surface area $S = k_S a^2$.\\
Circumradius $R = k_R a$.\\
Inradius $r = k_r a$.\\

Tetrahedron ($v = 4, e = 6$)
\begin{align*} 
  k_V &= \frac{\sqrt{2}}{12} & k_S &= \sqrt{3}\\
  k_R &= \frac{\sqrt{6}}{4}    & k_r &= \frac{\sqrt{6}}{12}
\end{align*}
Octahedron ($v = 6, e = 12$)
\begin{align*} 
  k_V &= \frac{\sqrt{2}}{3} & k_S &= 2\sqrt{3}\\
  k_R &= \frac{1}{\sqrt{2}}    & k_r &= \frac{1}{\sqrt{6}}
\end{align*}
Dodecahedron ($v = 20, e = 30$)
\begin{align*} 
  k_V &= \frac{15 + 7\sqrt{5}}{4} & k_S &= 3\sqrt{5(5+2\sqrt{5})}\\
  k_R &= \frac{(1+\sqrt{5})\sqrt{3}}{4} & k_r &= \frac{\sqrt{10 + 22/\sqrt{5}}}{4}
\end{align*}
Icosahedron ($v = 12, e = 30$)
\begin{align*}
  k_V &= \frac{5(3 + \sqrt{5})}{12} & k_S &= 5\sqrt{3}\\
  k_R &= \frac{\sqrt{2(5+\sqrt{5})}}{4} & k_r &= \frac{1}{2}\sqrt{\frac{7+3\sqrt{5}}{6}}
\end{align*}
\end{algorithm}

\problemtitle{Derivatives/Integrals}

\begin{align*}
\arcsin x & \rightarrow \frac{1}{\sqrt{1-x^2}} & 
\arccos x & \rightarrow -\frac{1}{\sqrt{1-x^2}} \\
\tan x & \rightarrow 1 + \tan^2 x &
\arctan x & \rightarrow \frac{1}{1+x^2}\\
\int \tan ax & = - \frac{\ln | \cos ax |}{a} &
\int x \sin ax & = \frac{\sin ax - ax \cos ax}{a^2}
\end{align*}
  
