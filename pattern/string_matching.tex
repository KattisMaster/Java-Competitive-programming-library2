\problemtitle{String Matching}

\begin{algorithm}{Knuth-Morrison-Pratt}
\reflisting{kmp}
\end{algorithm}

\begin{algorithm}{Regular expressions}
\reflisting{regexp}

\usage{ \sourceline{NFA n = parseRegExp(\"{}(abc$|${}def)*{}\"{});\\
   char* s = ...;\\
   size\_t match = n.match(s);}}
\characteristics{{\tt RegExp::match(string::iterator s)} returns the length of the longest string beginning in {\tt s} which matches the regexp, or {\tt string::npos} if there is no string beginning in {\tt s} which matches the regexp.}

The supported regexp constructions are
\begin{itemize}
 \item{{\tt a}}, where {\tt a} is any non-special (like {\tt *} for instance) character, matches {\tt a} exactly.
 \item{{\tt R*}}, where {\tt R} is a regexp, is the Kleene closure.
 \item{{\tt R1|R2}}, where {\tt R1} and {\tt R2} are regexps, is concatenation.
 \item{{\tt (R)}}, where {\tt R} is a regexp, is the grouping operation.
 \item{{\tt [aX]}}, where {\tt a} is any character $\neq$ {\tt ]} and {\tt X} is a sequence of characters $\neq$ {\tt ]}, is equivalent to {\tt a|[X]}.
 \item{{\tt \"{}X\"{}}}, where {\tt X} is a sequence of characters $\neq$ {\tt \"{}}, matches the literal string {\tt X} exactly.
\end{itemize}

Something you should know is that operator priority is not
implemented. Therefore, the regexp {\tt abc\*} is interpreted as {\tt
(abc)*}, {\tt abc|def*} as {\tt abc|((def)\*)}, and {\tt
(mupp)*(abc|def)*} as {\tt ((mupp)*(abc|def))*}.

Any regexp {\tt R} of the form {\tt X(Y}, where {\tt Y} is a correctly
parenthesized regexp and {\tt X} is an arbitrary sequency of characters,
will be interpreted as the regexp {\tt X(Y)}. That is: missing right
parenthesis will be "added" to the end of the string.

Any regexp {\tt R} of the form {\tt X)Y}, where {\tt X} is a correctly
parenthesized regexp and {\tt Y} is an arbitrary sequence of
characters, will be interpreted as the regexp {\tt X}. That is: the
first inbalanced parenthesis will be considered the end of string.

\end{algorithm}