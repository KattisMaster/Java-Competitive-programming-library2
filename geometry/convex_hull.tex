\problemtitle{Convex Hull}

\begin{algorithm}{Graham scan}
\usage{\sourceline{it hull\_end = convex\_hull(p.begin(), p.end())}}
\desc
Swaps the points in {\tt p} so the hull points are in order at the
beginning.  {\bf NB!} Does not handle coinciding points.
\end{algorithm}

\codenc{convex hull}{hull/convex_hull}

\begin{algorithm}{Three dimensional hull}
\complexity{n^2}

\usage{{\tt convex\_hull\_space(points p, int n, list<ABC> \&trilist)}}

\desc
{\tt trilist} is a list of ABC-tripples of indices of vertices in the 3D
point vector {\tt p}.
\note{Requires the hull to have positive volume. Arbitrarily
triangulates the surface of the hull.}
\end{algorithm}

\codenc{convex hull space}{hull/convex_hull_space}

\begin{algorithm}{Point inside hull}
\complexity{\log(n)}
\usage{{\tt inside\_hull(hull p, int n, point t)}}

\desc
Determine whether a point {\tt t} lies inside the hull given by the
point vector {\tt p}. The hull should not contain colinear points. A hull with
2~points are ok. The result is given as: 1~inside, 0~onedge, -1~outside.
\end{algorithm}

\codenc{inside hull}{hull/inside_hull}

\begin{algorithm}{Hull diameter}
\complexity{n}
\usage{{\tt hull\_diameter2(hull p, int n, \&i1, \&i2)}}

\desc
Determine the points that are farthest apart in a hull.
{\tt i1, i2} will be the indices to those points after the call.
The squared distance is returned.
\end{algorithm}

\codenc{hull diameter}{hull/hull_diameter}

\begin{algorithm}{Minimum enclosing circle}
\complexity{n}
%\usage{\sourceline{bool mec(p, n, c, \&i1, \&i2, \&i3[, eps]);}}
%\usage{\sourceline{double mec(p, n, c[, eps]);}}

\desc
Fills in c with the centre point of the minimum circle, enclosing the
n point vector p. 
The first version fills in indices to the points
determining the circle, and returns whether the third index is used.
The second version returns the enclosing circle radius as a double.
Colinearity of a third point is determined by the eps limit.
\end{algorithm}

\codenc{mec}{hull/mec}

\begin{algorithm}{Line-hull intersect}
\complexity{\log(n)}
\usage{
{\tt line\_hull\_intersect(hull p, int n, point p1, point p2, \&s1, \&s2)}}

\desc
Determine the intersection points of a hull with a line.
{\tt p1, p2, s1, s2} will be the intersection points and indices to the hull
line segments that intersect after the call. Returns whether there is an
intersection.
\end{algorithm}

\codenc{line hull intersect}{hull/line_hull_intersect}
