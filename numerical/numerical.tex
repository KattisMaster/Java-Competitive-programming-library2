\categorytitle{Number Theory}
\categorycontents{}

\problemtitle{Divisibility}
\begin{algorithm}{GCD}
\reflisting{gcd}
\reflisting{gcd fast}
\usage{ \sourceline{d = gcd( a, b );} }
\complexity{ \log(b) }
\usage{ \sourceline{d = gcd\_fast(a, b);} }
\complexity{ \log(a) + \log(b) }
\characteristics{The \sourceline{gcd\_fast} routine is slightly more writing than the usual gcd routine, but can be useful where speed is {\em really essential}, as it only uses subtraction and shift operations instead of the evil modulo.}

\valladolid{202}
\end{algorithm}

\def\lcm{\mathrm{lcm}}
\begin{algorithm}{LCM}
\keyword{} $\lcm(a,b)=\frac{ab}{\gcd(a,b)}$
\end{algorithm}

\begin{algorithm}{Euclid}
\reflisting{euclid}
\usage{ \sourceline{d = euclid( a, b, \&x, \&y );} \quad %
	$x$, $y$ will satisfy $ax+by=d$ after the call. }
\complexity{ \log(b) }
\characteristics{$x$ and $y$ have the smallest absolute value}
\begin{example}
	euclid( 10, 6, x, y ) == 2, x == -1, y == 2
\end{example}
\valladolid{202}
\end{algorithm}

\begin{algorithm}{$\phi$-function}
\reflisting{phi}
\keyword{} $\phi(n)=\#\{d<n|\gcd(n,d)=1\}$
\keyword{} If $n=\prod p_i^{k_i}$,
$\phi(n)=n\prod\left(1-\frac{1}{p_i}\right)=\prod p_i^{k_i-1}(p_i-1)$
\complexity{ \sqrt{n} }
\end{algorithm}

\problemtitle{Primes}
\begin{algorithm}{Primes}
\reflisting{primes}
\characteristics{{\tt primes} calculates a vector of primes and has a
factor method which returns the smallest factor in the given integer.}
\usage{ \sourceline{primes<int> p; p.generate(1000); n = factor(10); n==2;} }
\end{algorithm}

\begin{algorithm}{Prime Sieve}
\reflisting{prime sieve}
\characteristics{{\tt prime\_sieve} calculates a bool-vector containing
whether an integer is prime. It is faster than {\tt primes} when dealing
with large numbers. Returns whether an integer is prime in constant
time.}
\usage{ \sourceline{prime\_sieve p(1000); p.isprime(10)==false;} }
\end{algorithm}\

\begin{algorithm}{Primes Many}
\reflisting{primes many simple}
\reflisting{primes many fast}
\characteristics{{\tt primes\_many\_simple} and {\tt primes\_many\_fast}
calculates a vector of primes and is suitable when dealing with lots of
primes (10000 or more). It calculates them by first sieving and then
putting them into an array.}
\usage{ \sourceline{primes\_many p(100000); p.primes[2] == 5;} }
\end{algorithm}

\begin{algorithm}{Miller-Rabin}
\reflisting{miller-rabin}
\complexity{ s \log(n) }
\characteristics{{\tt isprime\_rabin\_miller} is a probabilistic primality
test. It will never say that a prime is composite but may erroneous claim
that a composite number is prime. The argument {\tt s} is the number of
iterations to be used. The probability of a false answer is not more than
$2^{-s}$ so $s=50$ is more than enough for most applications. The integer
to be tested {\tt n} may be as large as $2^{63}-1$. }
\usage{ \sourceline{ isprime\_rabin\_miller( n, s ); } }
\end{algorithm}

\begin{algorithm}{Number of divisors}
\reflisting{ndivisors}
\characteristics{{\tt ndivisors} and {\tt ndivisors\_prob} calculates
the number of divisors. {\tt ndivisors} needs a prime-table with all primes
upto the square root of the number and {\tt ndivisors\_prob} only primes as
large as the third root of the number. The second algorithm uses a
probabilistic primality test (Rabin-Miller).}
\usage{ \sourceline{primes\_many\_fast p(100000);
	ndivisors\_prob(23424234234243, p.primes) == 8;} }
\end{algorithm}

\problemtitle{Josephus}
\begin{algorithm}{Josephus}
\reflisting{josephus}
\complexity{log_{\frac{k}{k-1}}(n)}

Josephus is the problem to determine which person remains when repeatedly
removing the $k$:th person from a total of $n$ persons (cyclic).
\end{algorithm}


\problemtitle{Random}
\begin{algorithm}{Pseudo random numbers}
\reflisting{pseudo}
\characteristics{{\tt pseudo} gives a pseudo-random integer in
$[0,2^{31}-1]$.}
\characteristics{{\tt ullpseudo} gives a pseudo-random integer in
$[0,2^{62}-1]$.}
\characteristics{{\tt fpseudo} gives a pseudo-random number in
$[0,1)$.}
\complexity{1}
\end{algorithm}

\problemtitle{Bigmod}
\begin{algorithm}{mulmod}
\reflisting{mulmod}
\characteristics{{\tt mulmod} calculates $ab\mbox{ mod } n$ in a way that
allows numbers as large as MAXINT/2 to be used (which would otherwise give
an overflow).}
\usage{ \sourceline{ mulmod(5678945893454353ULL, 2423948234343ULL,
	31231231231231231ULL) == 14355903581119892; } }
\end{algorithm}

\begin{algorithm}{exptmod}
\reflisting{exptmod}
\complexity{\log{b}\textrm{ ops}}
\characteristics{{\tt exptmod} calculates $a^b\mod n$ by repeated
squaring.}
\end{algorithm}

\begin{sourceslandscape}
\code{gcd}{gcd}
\code{gcd fast}{gcd_fast}
\code{euclid}{euclid}
\code{phi}{phi}
\code{josephus}{josephus}
\code{pseudo}{pseudo}
\code{primes}{primes}
\code{primes many simple}{primes_many_simple}
\code{primes many fast}{primes_many_fast}
\code{prime sieve}{prime_sieve}
\code{miller-rabin}{miller-rabin}
\code{ndivisors}{ndivisors}
\code{ndivisors prob}{ndivisors_prob}
\code{mulmod}{mulmod}
\code{exptmod}{exptmod}
\end{sourceslandscape}
