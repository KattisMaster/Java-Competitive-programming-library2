\categorytitle{Games}
\categorycontents{}

\problemtitle{Repetitive Asymmetric Games}

Repetitive asymmetric games are best solved by recursing backwards from
known winning \& losing positions.

\problemtitle{Combinatorial Games}

\begin{algorithm}{Impartial take-and-break games (NIM-like games)}
\desc
An impartial take-and-break game is a game where two players take
turns removing (indistinguisable) tokens from a set of some heaps of
tokens.  The player removing the last token (thus causing the next
player to be unable to move) is the winner.  The moves available are:
removing $x$ tokens from a heap (for some set of allowed $x$), and
splitting a heap of $n$ tokens into two heaps of $n_1$ and $n_2$
tokens where $n_1, n_2 < n$.  Because every move reduces a heap size
by at least 1, such games can never end in draw.  To find optimal
strategies, Grundy numbers (or nimbers) can be used.  The Grundy value
of a state $S = \{n\}$ is defined as $G(S) =
\mathrm{mex}\;S'$ where $S'$ runs over all successor states to $S$ and
$\mathrm{mex}$ is the minimal excluded (nonnegative) value.  The
Grundy value of $S =
\{n_1, n_2, \ldots n_k\}$ is defined as $\bigoplus_{i=1}^{k}
G(\{n_i\})$.  A state $S$ is winning iff $G(S) \ne 0$.

\end{algorithm}

\problemtitle{Card Games}

\begin{algorithm}{Poker Hands}
\reflisting{poker hands}
\usage{ \sourceline{int i = hand\_value(int hand[5]); string s = hand\_names[i];} }
\valladolid{131}
\desc
Cards are assumed to be integers from $0..51$ where $0$ is the ace of
the first color, $12$ the king of the first color, $13$ the ace of the
second color, and so on.  (Note: no distinction is made between two
hands of equal value (t.b.a.).)

\end{algorithm}


\begin{sourceslandscape}
\code{rep asymm}{rep_asymm}
\code{poker hands}{poker_hands}
\end{sourceslandscape}
