\problemtitle{Euler walk}

\begin{algorithm}{Euler walk}
\reflisting{euler walk}
\complexity{ E }
\usage{\sourceline{%
 euler\_walk( V \&edges, int start, list<int> \&path, bool cyclic ) }}

\desc
Find an eulerian walk in a directed graph, i.e. a walk traversing all
edges exactly once.

The algorithm \emph{assumes} that there exists an eulerian walk. If it
does not exists, it will return any maximal path, not neccessarily the
longest.

If the graph is not cyclic, the start node must be a node with
$\mathrm{deg}_{out}-\mathrm{deg}_{in} = 1$.

Set {\tt cyclic=true} if the path found must be cyclic, this is mostly
of internal use.

{\tt edges} is a vector/array with $V$ edge-containers. The
edge-containers should contain vertex-indices, and may contain
repeated indices (i.e. multiple edges). {\bf WARNING!} {\tt edges} is
modified and emptied by the algorithm.

{\tt path} should be empty prior to the call and contains the
euler-path given as \emph{vertex numbers}. The first vertex is {\tt
start} which also is the last vertex if the path is cyclic.

\item[Lexicographic Path] If the edges are sorted in lexicographic order for
each vertex, the resulting path will be lexicographically
ordered. 
\end{algorithm}

\begin{algorithm}{Chinese postman}
\desc
A generalised euler path/cycle problem, finding the shortest
path/cycle that visits all edges even if some edges have to be
traversed several times.  There are several variations to this
problem, e.g. for directed or undirected graphs, paths or cycles, or
whether just a subset of the edges are interesting (the latter
variations are called rural chinese postman, and are generally
NP-complete).

Undirected chinese postman can be solved by computing a minimum
weighted matching on the odd nodes of the graph (described in
e.g. Edmonds and Johnson, ``Matching, Euler Tours and the Chinese
Postman.'', Mathematical Programming 5: 88-124, 1973).

Directed chinese postman can be solved by using network flow
techniques (also described in the previous reference).
\end{algorithm}

\problemtitle{De Bruijn Sequences}
Let $\Omega$ be an alphabet of size $\sigma$. A de Bruijn sequence is
a sequence such that all words on $L$ letters appear as a contiguous
subrange of it. In a cyclic de Bruijn sequences a word may also wrap
around the string.  The shortest cyclic de Bruijn sequence is of
length $\sigma^L$ and the shortest non-cyclic de Bruijn sequence is of
length $\sigma^L+L-1$.

The shortest de Bruijn sequence of all words on $3$ letters in the
alphabet $\{0,1\}$ which is lexicographically smallest is

$00011101$ (cyclic)

$0001110100$ (non-cyclic)

\begin{algorithm}{de Bruijn}
\reflisting{deBruijn}
\reflisting{deBruijn fast}
\complexity{ N^L }
\usage{\sourceline{%
 deBruijn( int N, int L, char symbols[N] ) }}

\desc
{\tt N} is the size of the alphabet and {\tt symbols} the
corresponding letters. {\tt L} is the length of the words that should
appear in the de Bruijn sequence.

The output is given as {\tt cout}-statements.
\end{algorithm}
