\problemtitle{Network Flow}


\code{flow graph}{maxflow/flow_graph}

\begin{algorithm}{Lift to Front}
\complexity{V^3}
\end{algorithm}
\label{liftofront}
\codenc{lift to front}{maxflow/lift_to_front}

\begin{algorithm}{Ford Fulkerson}
\desc
DFS and BFS Ford Fulkerson implementations.
\complexity{E\cdot n_{aug~paths}}
\end{algorithm}
\codenc{ford fulkerson}{maxflow/ford_fulkerson}

\begin{algorithm}{Flow constructions}
\keyword{Minimal cut} 
of a graph, generalization of edge connectivity. A minimal cut is
found by first finding a maximal flow. Then we consider the set $A$ of
all nodes that can be reached from the source using edges which has
capacity left (i.e. edges in the residue network). The edges between
$A$ and the complement of $A$ is a minimal cut.

\keyword{Minimal path cover} 
of a DAG, determines a minimum set of disjoint paths to cover all
vertices.  For each $i \in V$, create two new vertices $x_i$ and $y_i$
and draw an edge between $x_i$ and $y_j$ if $(i,j) \in E$.  Find a
maximal bipartite matching, and if $(x_i, y_j)$ is in the matching,
let the edge $(i,j)$ be part of a path.

\end{algorithm}
