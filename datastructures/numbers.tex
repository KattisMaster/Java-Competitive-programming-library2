%%%
%%% Numbers
%%%
\problemtitle{Numerical data structures}

\begin{algorithm}{Complex}
\usage{{\tt \#include <complex>}}
\end{algorithm}

\begin{algorithm}{Sign}
\reflisting{sign}
\end{algorithm}

\begin{algorithm}{Rational}
\reflisting{rational}
\end{algorithm}

\begin{algorithm}{Bigint Simple}
\reflisting{bigint simple}
\characteristics{Fully dynamic BigInt class which handles +,-,* for
positive integers. Can output numbers in base-10 only. This is a stripped
down version of {\tt bigint full}.}
\end{algorithm}

\begin{algorithm}{Bigint Full}
\reflisting{bigint full}
\characteristics{Fully dynamic BigInt class which handles +,-,*,/ for
positive integers. Can output numbers in base-10 only. Division/modulus is
simple and uses neither an iterative method, such as Newton-Raphson,
nor FFT. Has an iterative sqrt-function.}
\end{algorithm}

\begin{algorithm}{Bigint Per}
\reflisting{bigint per}
\characteristics{Dynamic BigInt class which handles +,*. Additionally
it can do divison/modulus with integers, find the n:th root of a number and
do exponentiation with \^~. NB! no subtraction.}
\end{algorithm}






