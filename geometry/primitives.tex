\problemtitle{Geometric primitives}
\begin{algorithm}{$\pi$, sqr, point and line}
\reflisting{geometry}

The recomended way of defining pi in contests. \\
The useful {\tt sqr} function. \\
A very simple point struct.
Normally one should use the one in {\tt point.cpp}. \\
A line struct.
\end{algorithm}

\begin{algorithm}{Point}
\reflisting{point}

A point struct with comparison, difference, inverse scaling, dot and
scalar cross product.
\end{algorithm}

\begin{algorithm}{Point operations}
\reflisting{point ops}

The operations {\tt dist2, dx, dy, dz, dist, angle, theta, unit, perm, normal}
on points. Theta is a rational meta-angle function in the rangle $(-2,2]$.
\end{algorithm}

\begin{algorithm}{Point 3D}
\reflisting{point3}

A 3D point struct with comparison, difference, inverse scaling, dot and
vector cross product.
\end{algorithm}

\begin{algorithm}{Inside polygon}
\reflisting{inside}

Determine whether a point is inside a polygon. If it is on an edge,
standard computer graphics rules determine the returned value (inside above
and to the left of the polygon, but not below or to the right).
This is usually \emph{not} the desired behaviour in contest geometry problems.
\end{algorithm}

\begin{algorithm}{Winding number}
\reflisting{winding number}

Theta meta-angle winding number of a point.
\end{algorithm}

\begin{algorithm}{Point-line relations}
\reflisting{point line}

Determines which side of a line a point is and whether a point is on a line
segment.
\end{algorithm}

\begin{algorithm}{Line intersection}
\reflisting{line isect}

Intersection point between two lines.
\end{algorithm}

\begin{algorithm}{Interval intersection}
\reflisting{interval}

Intersection between two intervals or rectangles.
\end{algorithm}

\begin{algorithm}{Polygon area}
\reflisting{poly area}
\reflisting{poly area too}

The signed area of a polygon calculated by adding cross product or
trapezium areas.
\end{algorithm}

\begin{algorithm}{Polyhedron volume}
\reflisting{poly volume}

The signed volume of a polyhedron calculated by adding vector tripple
product volumes.
\end{algorithm}

\begin{algorithm}{Polygon cut}
\reflisting{poly cut}
\usage{\sourceline{%
iterator r\_end = poly\_cut(v.begin(), v.end(), p0, p1, r.begin())}}

Cuts a polygon with (a half plane specified by) a line.
{\tt r} is filled in with the cut polygon, and the end of the filled in
interval is returned. The polygon is kept connected by (overlapping)
line segments along the cutting line if the cut splits the polygon in parts.
\end{algorithm}

\begin{algorithm}{Center of mass}
\reflisting{center of mass}

Polygon center of mass.
\end{algorithm}

\begin{algorithm}{Counter-Clock-Wise}
\reflisting{ccw}

Sedgewick ccw function.
\end{algorithm}

\begin{algorithm}{Line segment intersection test}
\reflisting{isect test}

Based on Sedgewick's ccw function.
\end{algorithm}
