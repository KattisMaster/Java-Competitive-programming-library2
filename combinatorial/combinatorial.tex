\categorytitle{Combinatorial}
\categorycontents{}


\problemtitle{Combinatorial Games}

\begin{algorithm}{Impartial take-and-break games (NIM-like games)}
\desc
An impartial take-and-break game is a game where two players take
turns removing (indistinguisable) tokens from a set of some heaps of
tokens.  The player removing the last token (thus causing the next
player to be unable to move) is the winner.  The moves available are:
removing $x$ tokens from a heap (for some set of allowed $x$), and
splitting a heap of $n$ tokens into two heaps of $n_1$ and $n_2$
tokens where $n_1, n_2 < n$.  Because every move reduces a heap size
by at least 1, such games can never end in draw.  To find optimal
strategies, Grundy numbers (or nimbers) can be used.  The Grundy value
of a state $S = \{n\}$ is defined as $G(S) =
\mathrm{mex}\;S'$ where $S'$ runs over all successor states to $S$ and
$\mathrm{mex}$ is the minimal excluded (nonnegative) value.  The
Grundy value of $S =
\{n_1, n_2, \ldots n_k\}$ is defined as $\bigoplus_{i=1}^{k}
G(\{n_i\})$.  A state $S$ is winning iff $G(S) \ne 0$.

\end{algorithm}


\problemtitle{Permutations}
\begin{algorithm}{Permutations to/from integers}
\reflisting{intperm}
\usage{{\tt int perm[n], x;\\perm\_to\_int(x, perm, perm + n);\\int\_to\_perm(x, perm, perm + n);}}
\complexity{ n^2 }
\desc
{\tt int\_to\_perm} and {\tt perm\_to\_int} defines a bijection
between permutations $\pi \in S_n$ and integers $x \in \{0, \ldots
n!-1\}$, with the additional property that it is order-preserving (so
that it could be used for generating all permutations in lexicographic
order, though {\tt next\_permutation} is much handier for this).  In
theory, this could be useful when doing some graph search on
permutations (though in practice, the number of integers for which
$n!$ but not $n^n$ memory is feasible is quite small (2-3 depending on
what you consider feasible)).
\end{algorithm}

\xinput{counting}

\begin{sourceslandscape}

\sourcesection{Permutations \& Counting}
\code{intperm}{intperm}
\code{choose}{counting/choose}
\code{multinomial}{counting/multinomial}
\code{stirling1}{counting/stirling1}
\code{stirling}{counting/stirling}
\code{stirling mod 2}{counting/smod2}
\code{euler}{counting/euler}
\code{euler2}{counting/euler2}
\end{sourceslandscape}
