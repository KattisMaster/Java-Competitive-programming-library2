\problemtitle{Shortest Path}

All algorithms work with a (directed) graph with non-negative distances.
Self and multiple edges are allowed. Floyd Warshall calculates the distances
between all pairs of nodes. The dijkstra variants calculate the
distance from a source to all other nodes.

The graph is given as a STL-container of
{\tt vector}s/{\tt map}s/{\tt multimap}s
(or {\tt set} in {\tt dijkstra\_1}) with edges.
The edges are simple indices for {\tt dijkstra\_1} and {\tt pair<int, T>} for
{\tt dijkstra} and {\tt dijkstra\_prim}.

The shortest distance to each node from the source is returned in a vector,
{\tt min} which must be pre-allocated to the same size as the number of nodes.
The path is similary returned in a pre-allocated vector. For each {\tt node},
{\tt path[node]} is the previous node from which the {\tt node} was entered
from in the shortest path to {\tt node}.

All graphs also take a number of nodes, {\tt n}, which usually is
{\tt edges.size()}.

\begin{algorithm}{Floyd Warshall}
\reflisting{floyd warshall}
\complexity{V^3}
\usage{ \sourceline{ floyd\_warshall(adj, path,n); } }

Floyd-Warshall uses an \emph{adjacency matrix} instead of a usual edge-list
graph. The adjacency matrix should contain the edge-cost between
{\tt a} and {\tt b} in {\tt adj[a][b]} if there is an edge between
{\tt a} and {\tt b} and otherwise {\tt adj[a][b]} should be negative.
The adjacency matrix is updated so the shortest path from {\tt a} to
{\tt b} is {\tt adj[a][b]} or $-1$ if no path exists.
\end{algorithm}

\begin{algorithm}{Dijkstra simple}
\reflisting{dijkstra simple}
\complexity{V^2+E}
\usage{ \sourceline{ dijkstra\_1(edges, min, path, start, n); } }
\end{algorithm}

\begin{algorithm}{Dijkstra 1}
\reflisting{dijkstra 1}
\complexity{V+E}
\usage{ \sourceline{ dijkstra\_1(edges, min, path, start, n); } }
\characteristics{{\tt dijsktra\_1} calcs the shortest path where all edges have
weight 1.}
\end{algorithm}

\begin{algorithm}{Dijkstra Prim}
\reflisting{dijkstra prim}
\complexity{V\log V+E}
\usage{ \sourceline{ dijkstra\_1(edges, min, path, start, n, distfun, mst); } }

{\tt dijkstra\_prim} takes two extra arguments, {\tt distfun} and {\tt mst}.
\end{algorithm}
