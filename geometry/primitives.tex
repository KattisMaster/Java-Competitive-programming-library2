\problemtitle{Geometric primitives}

%\begin{algorithm}{$\pi$, sqr}
%\desc
%The recomended way of defining pi in contests, and the useful {\tt sqr} function.
%\end{algorithm}
%\code{geometry}{geometry.h}


\code{point}{point}
\code{point3}{point3}
\code{point line relations}{point_line_relations}


\begin{algorithm}{Line intersection}
\desc
Intersection point between two lines.
\end{algorithm}
\code{line isect}{line_isect}

\begin{algorithm}{Interval union}
\desc
The union of several intervals given as pair<first,last> in a container. The
result is a disjoint list of intervals in ascending order.
\end{algorithm}
\code{ival union}{ival_union}


\begin{algorithm}{Circle tangents}
\desc
The tangent points from a point to a circle. The algorithm returns if
the point lies on the circles perimeter (in which case the two tangent
points are equal).
\end{algorithm}
\code{circle tangents}{circle_tangents}


\begin{algorithm}{Counter-Clock-Wise}
\desc
Sedgewick ccw function.
\end{algorithm}
\code{ccw}{ccw}

\begin{algorithm}{CCW Line segment intersection test}
\desc
Based on Sedgewick's ccw function.
\end{algorithm}
\code{isect test}{isect_test}

