%%%
%%% Number theory
%%%
\problemtitle{Number Theory}

Miscellaneous algorithms in number theory.

\begin{algorithm}{GCD}
\reflisting{gcd}
\usage{ \sourceline{d = gcd( a, b );} \quad\quad $a>b$ }
\complexity{ \log(b) }
\valladolid{202}
\end{algorithm}

\def\lcm{\mathrm{lcm}}
\begin{algorithm}{LCM}
\keyword{} $\lcm(a,b)=\frac{ab}{\gcd(a,b)}$
\end{algorithm}

\begin{algorithm}{Euclid}
\reflisting{euclid}
\usage{ \sourceline{d = euclid( a, b, \&x, \&y );} \quad $a>b$, %
	$x$ and $y$ are return values that satisfy $ax+by=d$. }
\complexity{ \log(b) }
\characteristics{$x$ and $y$ have (hopefully, probably, don't know...) the smallest absolute value}
\begin{example}
	euclid( 10, 6, x, y ) == 2, x == -1, y == 2
\end{example}
\valladolid{202}
\end{algorithm}

\begin{algorithm}{$\phi$-function}
\reflisting{phi}
\keyword{} $\phi(n)=\#\{d<n|\gcd(n,d)=1\}$
\keyword{} If $n=\prod p_i^{k_i}$,
$\phi(n)=n\prod\left(1-\frac{1}{p_i}\right)=\prod p_i^{k_i-1}(p_i-1)$
\complexity{ \mathcal{O}(\sqrt(n)) }
\end{algorithm}

\begin{algorithm}{Primes}
\reflisting{primes}
\characteristics{{\tt primes} calculates a vector of primes and has a
factor method which returns the smallest factor in the given integer.}
\usage{ \sourceline{primes<int> p; p.generate(1000); n = factor(10); n==2;} }
\end{algorithm}

\begin{algorithm}{Prime Sieve}
\reflisting{prime sieve}
\characteristics{{\tt prime\_sieve} calculates a bool-vector containing
whether an integer is prime. It is faster than {\tt primes} when dealing
with large numbers. Returns whether an integer is prime in constant
time.}
\usage{ \sourceline{prime\_sieve p(1000); p.isprime(10)==false;} }
\end{algorithm}\


\begin{algorithm}{Josephus}
\reflisting{josephus}
\complexity{log_{\frac{k}{k-1}}(n)}

Josephus is the problem to determine which person remains when repeatedly
removing the $k$:th person from a total of $n$ persons (cyclic).
\end{algorithm}


\begin{algorithm}{Pseudo random numbers}
\reflisting{pseudo}
\characteristics{{\tt pseudo} gives a pseudo-random integer in
$[0,2^{31}-1]$.}
\characteristics{{\tt fpseudo} gives a pseudo-random number in
$[0,1)$.}
\complexity{1}
\end{algorithm}
