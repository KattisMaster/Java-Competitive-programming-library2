\problemtitle{Convex Hull}

\paragraph{NOTE} None of the Graham scans handle multiple coinciding points,
so make sure the points are unique before calling!

\begin{algorithm}{Graham scan}
\reflisting{convex hull}
\complexity{n\log(n)}
\usage{\sourceline{iterator hull\_end = convex\_hull(p.begin(), p.end())}}

Swaps the points in {\tt p} so the hull points are in order at the beginning.
\characteristics{Handles colinear points on the hull as well}
\end{algorithm}

\begin{algorithm}{Graham scan, indexed}
\reflisting{convex hull idx}
\complexity{n\log(n)}
\usage{{\tt convex\_hull\_idx(points p, idx, int n)}}

Fills the index vector {\tt idx} with indeces to the point vector {\tt p},
returning the number of points in the hull.
\characteristics{Does not handle colinear points on the hull consistently}
\end{algorithm}

\begin{algorithm}{Graham scan, colinearly robust, indexed}
\reflisting{convex hull robust idx}
\complexity{n\log(n)}

\characteristics{Does handle colinear points on the hull consistently}
\end{algorithm}

\begin{algorithm}{Three dimensional hull}
\complexity{n^2}
\reflisting{convex hull space}

\usage{{\tt convex\_hull\_space(points p, int n, list<ABC> \&trilist)}}

{\tt trilist} is a list of ABC-tripples of indices of vertices in the 3D
point vector {\tt p}.
\characteristics{Requires the hull to have positive volume. Arbitrarily
triangulates the surface of the hull.}
\end{algorithm}

\begin{algorithm}{Point inside hull}
\reflisting{inside hull}
\complexity{\log(n)}
\usage{{\tt inside\_hull(hull p, int n, point t)}}

Determine whether a point {\tt t} lies inside the hull given by the
point vector {\tt p}.
\end{algorithm}

\begin{algorithm}{Hull diameter}
\reflisting{hull diameter}
\complexity{n}
\usage{{\tt hull\_diameter2(hull p, int n, \&i1, \&i2)}}

Determine the points that are farthest apart in a hull.
{\tt i1, i2} will be the indices to those points after the call.
The squared distance is returned.
\end{algorithm}

\begin{algorithm}{Minimum enclosing circle}
\reflisting{mec}
\complexity{n}
\usage{\sourceline{bool mec(p, n, c, \&i1, \&i2, \&i3[, eps]);}}
\usage{\sourceline{double mec(p, n, c[, eps]);}}

Fills in c with the centre point of the minimum circle, enclosing the
n point vector p. The first version fills in indices to the points
determining the circle, and returns whether the third index is used.
The second version returns the enclosing circle radius as a double.
Colinearity of a third point is determined by the eps limit.
\end{algorithm}

\begin{algorithm}{Line-hull intersect}
\reflisting{line hull intersect}
\complexity{\log(n)}
\usage{
{\tt line\_hull\_intersect(hull p, int n, point p1, point p2, \&s1, \&s2)}}

Determine the intersection points of a hull with a line.
{\tt p1, p2, s1, s2} will be the intersection points and indices to the hull
line segments that intersect after the call. Returns whether there is an
intersection.
\end{algorithm}
