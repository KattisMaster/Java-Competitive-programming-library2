%%%
%%% Combinatorics
%%%

\problemtitle{Counting}

\begin{algorithm}{Binomial $\binom{n}{k}$}
\reflisting{choose}
\complexity{\min\{k, n-k\}}
\end{algorithm}

\begin{algorithm}{Multinomial $\binom{\Sigma k_i}{k_1\;k_2\;\ldots\;k_n}$}
\reflisting{multinomial}
\complexity{(\Sigma k_i)-k_1}
\end{algorithm}

\begin{algorithm}{String permutations (multinomial)}
\reflisting{nperms}
\usage{ \sourceline{ string s; int n = n\_perms( s ); } }

Algorithm for calculating the number of permutations of a string
(multinomial numbers).
\end{algorithm}

\begin{algorithm}{Stirling numbers of the first kind}
\reflisting{stirling1}
\usage{ \sourceline{s = stirling1(n,k);} }

The Stirling numbers of the first kind $s_{nk}$ count the number of ways to permute a list of $n$ items into $k$ cycles. 
\end{algorithm}

\begin{algorithm}{Stirling numbers of the second kind}
\reflisting{stirling}
\usage{ \sourceline{s = stirling(n,k);} }

Calculates the stirling number $s_{nk}$, i.e. in how many ways can $n$
different items be put in $k$ boxes with at least one item in every box, or
mathematically speaking -- the number of partitions of $n$ elements into
$k$ partitions.
\end{algorithm}

\begin{algorithm}{Bell numbers}
\keyword{} $B(n) = \sum_{k=1}^n S(n,k)$, where S(n, k) is the Sterling numbers of the second kind.

The Bell number count the ways $n$ elements can be partitioned.
\end{algorithm}

\begin{algorithm}{Catalan numbers}
\keyword{}
Among other things, the Catalan numbers describe the number of ways a polygon
with n+2 sides can be cut into n triangles, the number of ways in which
parentheses can be placed in a sequence of numbers to be multiplied, two at
a time; the number of rooted, trivalent trees with n+1 nodes; and the number
of paths of length 2n through an n-by-n grid that do not rise above the
main diagonal.
$$ C_n = \frac{\binom{2n}{n}}{\scriptstyle n+1} $$
(from Math forum)
\end{algorithm}

\begin{algorithm}{Derangements}
\keyword{}
A permutation that leaves no element in its original position.
$$D_{n+1}=n(D_n+D_{n-1})
\qquad
D_n=n!\left(\frac 1{2!}-\frac 1{3!}+\ldots+(-1)^n\frac 1{n!}\right), n\ge 2$$
\end{algorithm}


