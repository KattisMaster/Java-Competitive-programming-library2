%%%
%%% Numbers
%%%
\problemtitle{Numerical data structures}

\begin{algorithm}{Complex}
\usage{{\tt \#include <complex>}}
\end{algorithm}

\begin{algorithm}{Sign}
\reflisting{sign}
\desc
Wrapper class to add a sign to an unsigned number type (i.e. typically
a bigint).
\end{algorithm}

\begin{algorithm}{Rational}
\reflisting{rational}
\desc
Class for rational numbers over a number type (i.e. typically some
integer type).
\end{algorithm}

\begin{algorithm}{Bigint}
\reflisting{bigint}
\end{algorithm}

\begin{algorithm}{Bigint Simple}
\reflisting{bigint simple}
\desc
Fully dynamic BigInt class which handles +,-,* for
positive integers. Can output numbers in base-10 only. This is a stripped
down version of {\tt bigint full}.
\end{algorithm}

\begin{algorithm}{Bigint Full}
\reflisting{bigint full}
\desc
Fully dynamic BigInt class which handles +,-,*,/ for
positive integers. Can output numbers in base-10 only. Division/modulus is
simple and uses neither an iterative method, such as Newton-Raphson,
nor FFT. Has an iterative sqrt-function.
\end{algorithm}

\begin{algorithm}{Bigint Per}
\reflisting{bigint per}
\desc
Fully dynamic (unsigned) BigInt class which handles
+,-,*,/. Additionally it can find the $n$:th root of a number and do
exponentiation with \^~. Input and output in base 10.
\end{algorithm}
