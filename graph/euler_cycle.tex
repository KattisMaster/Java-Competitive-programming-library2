\problemtitle{Euler walk}

\begin{algorithm}{Euler walk}
\complexity{ E }

\desc
The algorithm \emph{assumes} that there exists an eulerian walk. If it
does not exists, it will return any maximal path, not neccessarily the
longest.  If the graph is not cyclic, the start node must be a node
with $\mathrm{deg}_{out}-\mathrm{deg}_{in} = 1$.  Set {\tt
cyclic=true} if the path found must be cyclic, this is mostly of
internal use.

{\tt edges} is a vector/array with $V$ edge-containers. The
edge-containers should contain vertex-indices, and may contain
repeated indices (i.e. multiple edges). {\bf WARNING!} {\tt edges} is
modified and emptied by the algorithm.

{\tt path} should be empty prior to the call and contains the
euler-path given as \emph{vertex numbers}. The first vertex is {\tt
start} which also is the last vertex if the path is cyclic.

\item[Lexicographic Path] If the edges are sorted in lexicographic order for
each vertex, the resulting path will be lexicographically
ordered. 
\end{algorithm}

\codenc{euler walk}{euler_walk}

\begin{algorithm}{Chinese postman}
\desc
A generalised euler path/cycle problem, finding the shortest
path/cycle that visits all edges even if some edges have to be
traversed several times.  There are several variations to this
problem, e.g. for directed or undirected graphs, paths or cycles, or
whether just a subset of the edges are interesting (the latter
variations are called rural chinese postman, and are generally
NP-complete).

Undirected chinese postman can be solved by computing a minimum
weight perfect matching on the odd nodes of the graph.% (described in
%e.g. Edmonds and Johnson, ``Matching, Euler Tours and the Chinese
%Postman.'', Mathematical Programming 5: 88-124, 1973).

Directed chinese postman can be solved by computing a minimum weight
perfect bipartite matching between the positive and negative nodes of
the graph.  If a node $v$ has surplus (outvalency - invalency) $k$,
$k$ copies of this node occurs in the bipartite graph. {\bf NB!}  The
bipartite graph may have $2E$ vertices!
\end{algorithm}

\begin{algorithm}{De Bruijn Sequences}
\complexity{ N^L }
\usage{\sourceline{%
 deBruijn( int N, int L, char symbols[N] ) }}

\desc
Let $\Omega$ be an alphabet of size $\sigma$. A de Bruijn sequence is
a sequence such that all words on $L$ letters appear as a contiguous
subrange of it.  The shortest cyclic de Bruijn sequence is of
length $\sigma^L$ and the shortest non-cyclic de Bruijn sequence is of
length $\sigma^L+L-1$.

The lexicographically smallest, and shortest de Bruijn sequence on
three letters and alphabet $\{0,1\}$ is $00011101$ (cyclic),
$0001110100$ (non-cyclic).

{\tt N} is the size of the alphabet and {\tt symbols} the
corresponding letters. {\tt L} is the length of the words that should
appear in the de Bruijn sequence.

The answer is stored in the {\tt sol} string.
\end{algorithm}

% \code{deBruijn}{deBruijn}
\codenc{deBruijn fast}{deBruijn_fast}
