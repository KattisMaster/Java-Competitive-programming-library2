\problemtitle{Polygons}
\begin{algorithm}{Inside polygon}
\complexity{n}
\usage{\sourceline{ inside(polygon,nPts,point) == true; } }
\desc
Determine whether a point is inside a polygon. If it is on an edge,
standard computer graphics rules determine the returned value (inside above
and to the left of the polygon, but not below or to the right).
This is usually \emph{not} the desired behaviour in contest geometry problems.
Use {\tt on\_edge} in {\tt pointline.cpp} to check if a point is on the edge.
\end{algorithm}
\code{inside}{inside}


\begin{algorithm}{Winding number}
\complexity{n}

\desc
Theta meta-angle winding number of a point with a polygon. The polygon should
be given in ccw-order. Also function {\tt inside\_wn} which does the same as
{\tt inside} but uses the winding number. The value of {\tt inside\_wn} is
+1/-1 for inside/outside and 0 for on the edge.
\end{algorithm}
\code{winding number}{winding_number}

\begin{algorithm}{Polygon area}
\desc
Twice the signed polygon area.
\end{algorithm}
\code{poly area}{poly_area}

\begin{algorithm}{Polyhedron volume}
\desc
Signed polyhedron volume.
\end{algorithm}
\code{poly volume}{poly_volume}

\begin{algorithm}{Polygon cut}
\usage{\sourceline{%
iterator r\_end = poly\_cut(v.begin(), v.end(), p0, p1, r.begin())}}

\desc
Cuts a polygon with (a half plane specified by) a line.
{\tt r} is filled in with the cut polygon, and the end of the filled in
interval is returned. The polygon is kept connected by (overlapping)
line segments along the cutting line if the cut splits the polygon in parts.
\end{algorithm}

\code{poly cut}{poly_cut}


\begin{algorithm}{Center of mass}
\desc
Polygon and triangular center of mass.
\end{algorithm}
\code{center of mass}{centerofmass}

