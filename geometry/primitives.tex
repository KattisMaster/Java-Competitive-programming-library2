\problemtitle{Geometric primitives}
\begin{algorithm}{$\pi$, sqr, point and line}
\reflisting{geometry}

The recomended way of defining pi in contests. \\
The useful {\tt sqr} function. \\
A very simple point struct.
Normally one should use the one in {\tt point.cpp}. \\
A line struct.
\end{algorithm}

\begin{algorithm}{Point}
\reflisting{point}

A point struct with comparison, difference, inverse scaling, dot and
scalar cross product.
\end{algorithm}

\begin{algorithm}{Point operations}
\reflisting{point ops}

The operations {\tt dist2, dx, dy, dz, dist, angle, theta, unit, perm, normal}
on points. Theta is a rational meta-angle function in the rangle $(-2,2]$.
\end{algorithm}

\begin{algorithm}{Point 3D}
\reflisting{point3}

A 3D point struct with comparison, difference, inverse scaling, dot and
vector cross product.
\end{algorithm}

\begin{algorithm}{Point-line relations}
\reflisting{point line}
\item[{\tt sideof}] Determine on which side of a line a point is.
+1/-1 is left/right of vector $p_1-p_0$ and 0 is on the line.
\item[{\tt onsegment}] Determine if a point is on a line segment (incl the end
points).
\item[{\tt linedist}] Get a measure of the distance of a point from a line
(0 on the line and positive/negative on the different sides).
\end{algorithm}

\begin{algorithm}{Line intersection}
\reflisting{line isect}

Intersection point between two lines.
\end{algorithm}

\begin{algorithm}{Interval intersection}
\reflisting{interval}

Intersection between two intervals or rectangles.
\end{algorithm}

\begin{algorithm}{Interval intersection}
\reflisting{interval}

Intersection between two intervals or rectangles.
\end{algorithm}

\begin{algorithm}{Interval union}
\reflisting{ival union}

The union of several intervals given as pair<first,last> in a container. The
result is a disjoint list of intervals in ascending order.
\end{algorithm}

\begin{algorithm}{Circle tangents}
\reflisting{circle tangents}

The tangent points from a point to a circle. The algorithm returns if
the point lies on the circles perimeter (in which case the two tangent
points are equal).
\end{algorithm}

\begin{algorithm}{Counter-Clock-Wise}
\reflisting{ccw}

Sedgewick ccw function.
\end{algorithm}

\begin{algorithm}{CCW Line segment intersection test}
\reflisting{isect test}

Based on Sedgewick's ccw function.
\end{algorithm}
