%%%
%%% Number theory
%%%
\problemtitle{Number Theory}

Miscellaneous algorithms in number theory.

\begin{algorithm}{GCD}
\reflisting{gcd}
\usage{ \sourceline{d = gcd( a, b );} \quad\quad $a>b$ }
\complexity{ \log(b) }
\valladolid{202}
\end{algorithm}

\begin{algorithm}{LCM}
\complexity{x}
\end{algorithm}

\begin{algorithm}{Euclid}
\reflisting{euclid}
\usage{ \sourceline{d = euclid( a, b, \&x, \&y );} \quad $a>b$, %
	$x$ and $y$ are return values that satisfy $ax+by=d$. }
\complexity{ \log(b) }
\characteristics{$x$ and $y$ have (hopefully, probably, don't know...) the smallest absolute value}
\begin{example}
	euclid( 10, 6, x, y ) == 2, x == -1, y == 2
\end{example}
\valladolid{202}
\end{algorithm}


\begin{algorithm}{Josephus}
\reflisting{josephus}
\complexity{log_{\frac{k}{k-1}}(n)}

Josephus is the problem to determine which person remains when repeatedly
removing the $k$:th person from a total of $n$ persons (cyclic).
\end{algorithm}


\begin{algorithm}{pseudo}
\reflisting{pseudo}
\characteristics{{\tt pseudo} gives a pseudo-random integer in $[0,2^{31}-1]$.}
\complexity{1}
\end{algorithm}

\begin{algorithm}{fpseudo}
\reflisting{pseudo}
\characteristics{{\tt fpseudo} gives a pseudo-random number in $[0,1)$.}
\complexity{1}
\end{algorithm}
