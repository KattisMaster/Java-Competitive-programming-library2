\categorytitle{General}
\categorycontents{}

\problemtitle{Indexed}
\begin{algorithm}{Indexed}
  \usage{\sourceline{ isort( array, n, idxarray, comp) }}
  \reflisting{indexed}
\end{algorithm}

\problemtitle{Named items}
When items are named, for example graph nodes identified by strings or big
non-contiguous integers, it is often practical to keep a map from the
names to an index numbering starting from 0, and a vector to retrieve a name
back from an index.

The {\tt index mapper} does this.
It has function semantics to retrieve an index for a name,
and vector semantics to retrieve a name from an index:
\begin{algorithm}{Index mapper}
  \usage{\sourceline{%
int idx = mapper("x")  =>  mapper("x") == idx, mapper[idx] == "x"%
}}
  \reflisting{index mapper}
\end{algorithm}

\begin{sourceslandscape}
\code{indexed}{indexed}
\code{index mapper}{index_mapper}
\end{sourceslandscape}
