%%%
%%% Numbers
%%%
\problemtitle{Numerical data structures}

\begin{algorithm}{Complex}
\usage{{\tt \#include <complex>}}
\end{algorithm}

\begin{algorithm}{Sign}
\reflisting{sign}
\end{algorithm}

\begin{algorithm}{Rational}
\reflisting{rational}
\end{algorithm}

\begin{algorithm}{Bigint}
\reflisting{bigint}
\end{algorithm}

\begin{algorithm}{Bigint Simple}
\reflisting{bigint simple}
\desc
Fully dynamic BigInt class which handles +,-,* for
positive integers. Can output numbers in base-10 only. This is a stripped
down version of {\tt bigint full}.
\end{algorithm}

\begin{algorithm}{Bigint Full}
\reflisting{bigint full}
\desc
Fully dynamic BigInt class which handles +,-,*,/ for
positive integers. Can output numbers in base-10 only. Division/modulus is
simple and uses neither an iterative method, such as Newton-Raphson,
nor FFT. Has an iterative sqrt-function.
\end{algorithm}

\begin{algorithm}{Bigint Per}
\reflisting{bigint per}
\desc
Fully dynamic BigInt class which handles +,-,*. Additionally it can do
divison/modulus with \texttt{int}s, find the n:th root of a number and
do exponentiation with \^~. Input and output in base 10.
\end{algorithm}


\begin{algorithm}{Bigint Summary}
\desc
This table summarise the capabilities of the different Bigint
implementations.

\begin{tabular}{|c|c|c|c|c|}
\hline
                  & Bigint & BI Simple & BI Full & BI Per \\
\hline
Lines                        & 185 & 192 & 318 & 167 \\
\hline
Add, Sub, Cmp                & \multicolumn{4}{|c|}{\ordo{n} for all four} \\
\hline
Mul with limb                & \ordo{n} & N/A & N/A & \ordo{n} \\
\hline
Multiplication               & \multicolumn{4}{|c|}{\ordo{n^2} for all four} \\
\hline
DivMod with limb             & \ordo{n} & N/A & N/A & \ordo{n} \\
\hline
DivMod                       & \ordo{n^2} & N/A & yes & N/A \\
\hline
Exponentiation $N^e$         & N/A & N/A & N/A & \ordo{e \cdot n^2} \\
\hline
Square root $\sqrt{N}$       & N/A & N/A & yes & N/A \\
\hline
$e$th root $\sqrt[e]{N}$     & N/A & N/A & N/A & yes \\
\hline
I/O Base                     & Any, but fixed & 10 & 10 & 10\\
\hline
Bitops with limb             & \ordo{1} & N/A & N/A & N/A\\
\hline
gcd                          & N/A & N/A & yes & N/A \\
\hline
\end{tabular}


Here, $N$ is an $n$-bit number.
\end{algorithm}

