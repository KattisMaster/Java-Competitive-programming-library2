\categorytitle{Contest}
\categorycontents{}

\problemtitle{Practice session}

\begin{algorithm}{Checklist}
\reflisting{checklist}
\end{algorithm}

\begin{algorithm}{us\_key.modmap}
\reflisting{uskey}
\end{algorithm}

\problemtitle{Mandatory contest material}

\begin{algorithm}{Problem assessment sheet}
\usage{}
\end{algorithm}

\begin{algorithm}{Template}
\reflisting{Template}
\usage{}
Standard problem template. Problems are classified as either of:
  \begin{description}
  \item{simple-solve} -- Number of test cases is 1.
  \item{for-solve} -- Number of test cases is given in the input.
  \item{while-solve} -- End of test cases is indicated by special values
    or end of input. Stop when {\tt solve} returns false.
  \end{description}
\end{algorithm}

\begin{algorithm}{c-lite.el}
  \reflisting{c-lite.el}
  \usage{\sourceline{ M-x load-file RET c-lite.el }}

  c-lite.el provides useful key bindings for Emacs.
  \begin{description}
  \item{C-x C-f} -- New file (overloads the usual find-file, uses Template.cpp)
  \item{C-c C-c} -- Compile (uses compile command as specified in c-lite.el)
  \item{C-c C-t} -- Test solution (using ``\texttt{FILE < FILE.in}''. The commented line is for testing using ``\texttt{FILE}'', i.e. when files are used instead of stdio)
  \item{C-c C-s} -- Submit solution (using ``\texttt{submit FILE.cc}'')
  \end{description} 
\end{algorithm}

\begin{algorithm}{Script}
  \reflisting{script}
  \usage{\sourceline{ sh script }}
  \begin{description}
  \item{c} -- Compile
  \item{i} -- Enter program input
  \item{o} -- Enter correct program answer
  \item{t} -- Test using entered program input
  \item{td} -- Test with direct typed input
  \item{d} -- Diff the program output with the entered correct answer
  \item{p} -- Print source code
  \item{submit} -- Submit a solution!!
  \item{n} -- New problem, copy Template
  \item{f} -- Finished problem, move to done
  \end{description}
\end{algorithm}

\problemtitle{Optional contest material}

\begin{algorithm}{adler}
\reflisting{adler}
\usage{\sourceline{ adler < code.cpp }}

Adler gives a checksum for all non-whitespace character it reads.
All source code listings in this document have their checksum attached,
calculated after comments and preprocessor directives have been stripped off
(except for preprocessor directives in the code library utilities themselves).
\end{algorithm}

\begin{algorithm}{linecode}
\reflisting{linecode}
\usage{\sourceline{ linecode < code.cpp }}

Linecode gives 16 values encoding line number xor:s of each line's adler
checksum. All source code listings in this document have their line code
attached. If a checksum doesn't match for a typed-in file, xor the line code
from the document and the typed-in file to obtain candidate line numbers where
the error might be.
\end{algorithm}

\begin{algorithm}{xor}
\reflisting{xor}
-- A simple xor calculator.
\end{algorithm}

\begin{sourceslandscape}

\sourcesection{Practice Session}
\code{checklist}{checklist}
\code{uskey}{us_key.modmap}

\sourcesection{Mandatory Contest Material}
\code{Template}{Template}
\code{script}{script}
\code{adler}{../util/adler}

\sourcesection{Optional Contest Material}
\code{linecode}{../util/linecode}
\code{xor}{../util/xor}
\begin{verbatim}
xor table:
 0 1 2 3 4 5 6 7 8 9 a b c d e f
 1 0 3 2 5 4 7 6 9 8 b a d c f e
 2 3 0 1 6 7 4 5 a b 8 9 e f c d
 3 2 1 0 7 6 5 4 b a 9 8 f e d c
 4 5 6 7 0 1 2 3 c d e f 8 9 a b
 5 4 7 6 1 0 3 2 d c f e 9 8 b a
 6 7 4 5 2 3 0 1 e f c d a b 8 9
 7 6 5 4 3 2 1 0 f e d c b a 9 8
 8 9 a b c d e f 0 1 2 3 4 5 6 7
 9 8 b a d c f e 1 0 3 2 5 4 7 6
 a b 8 9 e f c d 2 3 0 1 6 7 4 5
 b a 9 8 f e d c 3 2 1 0 7 6 5 4
 c d e f 8 9 a b 4 5 6 7 0 1 2 3
 d c f e 9 8 b a 5 4 7 6 1 0 3 2
 e f c d a b 8 9 6 7 4 5 2 3 0 1
 f e d c b a 9 8 7 6 5 4 3 2 1 0
\end{verbatim}
\code{c-lite.el}{c-lite.el}
\end{sourceslandscape}
